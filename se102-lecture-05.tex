\documentclass[t]{beamer}
\usefonttheme{serif}

\usepackage{amsmath,amsthm,amssymb,amsfonts,amscd,mathrsfs,amsxtra,multirow,kotex,mathtools,gensymb,textcomp,lipsum,tikz,verbatim,color,soul,courier,mdframed,xcolor}
\usepackage[normalem]{ulem}
\usetikzlibrary{calc,matrix,arrows,chains,positioning,scopes}
\usepackage{pdfpages}

\theoremstyle{plain}
\newtheorem{thm}{Theorem}[section]
\newtheorem{prop}[thm]{Proposition}

\theoremstyle{definition}
\newtheorem{defn}[thm]{Definition}
\newtheorem{exmp}[thm]{Example}
\newtheorem{excs}[thm]{Exercise}
\newtheorem{rem}[thm]{Remark}
\newtheorem{prob}[thm]{Problem}
\newtheorem{cor}[thm]{Corollary}

\newcommand \tr[1]{\textcolor{red}{#1}}
\newcommand{\tikzmark}[1]{\tikz[overlay,remember picture] \node (#1) {};}
\newcommand{\varep}{\varepsilon}
\newcommand{\DrawBox}[1][]{%
    \tikz[overlay,remember picture]{
    \draw[red,#1]
      ($(left)+(-0.2em,0.9em)$) rectangle
      ($(right)+(0.2em,-0.3em)$);}
}

\newcommand{\tikzmarkk}[2]{
    \tikz[overlay,remember picture,baseline] 
    \node[anchor=base] (#1) {$#2$};
}
\newcommand*\circled[1]{\tikz[baseline=(char.base)]{
            \node[shape=circle,draw,inner sep=2pt] (char) {#1};}}

\tikzset{join/.code=\tikzset{after node path={%
\ifx\tikzchainprevious\pgfutil@empty\else(\tikzchainprevious)%
edge[every join]#1(\tikzchaincurrent)\fi}}}

\tikzset{>=stealth',every on chain/.append style={join},
         every join/.style={->}}
\tikzstyle{labeled}=[execute at begin node=$\scriptstyle,
   execute at end node=$]

\newenvironment<>{proofs}[1][\proofname]{%
   \par
   \def\insertproofname{#1\@{.}}%
   \usebeamertemplate{proof begin}#2}
 {\usebeamertemplate{proof end}}
 

\addtobeamertemplate{navigation symbols}{}{%
    \usebeamerfont{footline}%
    \usebeamercolor[fg]{footline}%
    \hspace{1em}%
    \raisebox{2pt}[0pt][0pt]{\insertframenumber/\inserttotalframenumber}
}
\setbeamercolor{footline}{fg=blue}
\setbeamerfont{footline}{series=\bfseries}
\title[]{SE102:Multivariable Calculus}

\author[]{Hyosang Kang\inst{1}}

\institute[]{\inst{1}Division of Mathematics\\ School of Interdisciplinary Studies\\ DGIST}

\date[]{Lecture 05\\
Multiple Integrals}

\begin{document}

\begin{frame}
\titlepage
\end{frame}

\begin{frame}
\begin{defn}[Double integral]
Let $f(x,y)$ be a function defined on a rectangular region $D=[a,b]\times[c,d]$.
Let us subdivide the intervals $[a,b]$ and $[c,d]$ by $n$ and $m$ subintervals:
	\[a=x_0<x_1<\cdots<x_n=b,\quad c=y_0<y_1<\cdots<y_m=d.\]
This subdivides $D$ into $nm$ subregions $D_{ij}=[x_{i-1},x_i]times [y_{i-1},y_i]$.
($i=1,\cdots,n$, $j =1,\cdots,m$) 
Denote $\Delta x_i=x_{i}-x_{i-1}$, $\quad\Delta y_j=y_{j}-y_{j-1}$.
The \textbf{Riemann sum} of $f(x,y)$ with respect to the subdivision of $D$ is
	\[\sum_{i=1}^n\sum_{j=1}^mf(x_i^*,y_j^*)\Delta x_i\Delta y_j\]
if $(x_i^*,y_j^*)\in D_{ij}$.  
\end{defn}
\end{frame}

\begin{frame}
If the limit exists when $n,m\to\infty$ regardless of choice of $x_i^*$ and $y_i^*$,
we say $f$ is \textbf{integrable} on $D$ and the limit is called the
\textbf{double integral}. We denote the limit as 
	\[\iint_DfdA=\lim\sum_{i=1}^n\sum_{j=1}^mf(x_i^*,y_i^*)\Delta x_i\Delta y_j.\]
\begin{thm}
If $f$ is continuous on the rectangular region $D=[a,b]\times[c,d]$, 
then $f$ is integrable on $D$. 
\end{thm}
\end{frame}

\begin{frame}
\begin{exmp}
Coumpute
	\[\iint_D xy dxdy\]
for $D = [0,1]\times[0,1]$.
\end{exmp}
\begin{exmp}
Show that the function $f(x,y)$ on $[0,1]\times[0,1]$ defined by
	$$f(x,y)=\begin{dcases}
	1 & x\textrm{ or }y\textrm{ is rational}\\
	0 & \textrm{otherwise}
	\end{dcases}$$
is not integrable on $[0,1]\times[0,1]$.
\end{exmp}
\end{frame}

\begin{frame}
\begin{defn}
Let $D\subset\mathbf R^2$ be bounded region contained in a rectangular box
$[a,b]\times[c,d]$. We say $f$ is integrable on $D$ if
the function $F$ below is integrable on $[a,b]\times[c,d]$.
\[F(x,y) = \begin{dcases}
	f(x,y) & (x,y)\in D \\
	0 & (x,y)\in[a,b]\times[c,d]\end{dcases}\] 
Moreover, the integral of $f$ on $D$ is defined by 
\[ \iint_D fdxdy = \iint\displaylimits_{[a,b]\times[c,d]} Fdxdy.\]
\end{defn}
\begin{thm}
Let $D$ be a bounded region. If a function $f(x,y)$ is continuous on $D$,
then $f$ is integrable.
\end{thm}
\end{frame}

\begin{frame}
\begin{defn}[Iterated integrals]
Let $f(x,y)$ be a two variable function define on a rectangular domain $D=[a,b]\times[c,d]$.
The \textbf{iterated integral} $\displaystyle \int_c^d\int_a^b f(x,y) dxdy$ on $D$ is defined as follows.
	$$\int_c^d\underbrace{\left[\int_a^b f(x,y)dx\right]}_{\textrm{consider $y$ as a constant}}dy.$$
\end{defn}
\begin{exmp}
Compute $\displaystyle \int_0^1\int_0^1\frac{y}{1+xy}dxdy$.
\end{exmp}
\end{frame}

\begin{frame}
\begin{thm}[Fubini I]
Let $f(x,y)$ be a continuous function defined on $D=[a,b]\times[c,d]$.
Then
	\[\iint_DfdA = \int_a^b \int_c^df(x,y)dy dx = \int_c^d \int_a^bf(x,y)dx dy\]
\end{thm}
\end{frame}

\begin{frame}
\begin{rem}
The \emph{continuity} condition in the theorem is crucial.
For example, let
	$$f(x,y)=\begin{dcases}
	\frac{x^2-y^2}{(x^2+y^2)^2}&(x,y)\neq(0,0)\\
	0&(x,y)=(0,0)
	\end{dcases}$$
Let us compute $\displaystyle\int_0^1\int_0^1f(x,y)dydx$ first.
	\begin{align*}
	\int_0^1\int_0^1\frac{x^2-y^2}{(x^2+y^2)^2}dydx &= \int_0^1 \frac{y}{x^2+y^2}\Big\vert_{y=0}^1dx \\
	&= \int_0^1 \frac{1}{1+x^2} dx = \tan^{-1}x\Big\vert_0^1 = \frac{\pi}{4}
    \end{align*}
\end{rem}
\end{frame}

\begin{frame}
Next, the iterated integral $\displaystyle\int_0^1\int_0^1f(x,y)dxdy$ is 
	\begin{align*}
	\int_0^1\int_0^1\frac{x^2-y^2}{(x^2+y^2)^2}dxdy &= \int_0^1 \frac{-x}{x^2+y^2}\Big\vert_{x=0}^1dx \\
	&= \int_0^1 \frac{-1}{1+y^2} dy = -\tan^{-1}y\Big\vert_0^1 = -\frac{\pi}{4}
	\end{align*}
and it does not coincide with $\displaystyle\int_0^1\int_0^1f(x,y)dydx$.
\end{frame}

% \begin{frame}
% \begin{defn}
% Let $f(x,y)$ be defined on a \underline{bounded} region $D$ in $\mathbf R^2$.
% Suppose that $D$ lies on a large rectangular domain, say $D\subset[a,b]\times[c,d]$. 
% Let us define a new function $F(x,y)$ as follows.
%     $$F(x,y)
%     = \begin{dcases}
% 	f(x,y) & (x,y)\in D\\
% 	0 & (x,y)\notin D
% 	\end{dcases}$$
% Then the \textbf{definite integral} of $f$ over the domain $D$ is defined as 
% 	\[\iint_Df(x,y)dxdy=\iint\displaylimits_{[a,b]\times[c,d]}F(x,y)dxdy\]
% \end{defn}
% \end{frame}

\begin{frame}
\begin{thm}[Fubini II]
Let  $f(x,y)$ be a continuous function defined on $D$.
If $D=\{(x,y)\,|\,a\le x\le b,\, g_1(x)\le y\le g_2(x)\}$,
then
	\[\iint_Df(x,y)dxdy=\int_a^b \int_{g_1(x)}^{g_2(x)}f(x,y)dy dx\]
Similarly, if $D=\{(x,y)\,|\,c\le y\le d,\, h_1(y)\le x\le h_2(y)\}$,
then
	\[\iint_Df(x,y)dxdy=\int_c^d \int_{h_1(y)}^{h_2(y)}f(x,y)dx dy\]
\end{thm}
\end{frame}

\begin{frame}
\begin{exmp} 
Let us compute $\displaystyle \iint_De^{-y^2}dxdy$ where $D$ is the triangular region whose vertices are
$(0,0)$, $(0,1)$, and $(1,1)$.
By Fubini's theorem,
	$$\iint_De^{-y^2}dxdy 
	= \int_0^1\left[\int_x^1 e^{-y^2}dy\right]dx
	= \int_0^1\left[\int_0^y e^{-y^2}dx\right]dy.$$
Find which integration works.
\end{exmp}
\end{frame}

\begin{frame}
\begin{defn}[Multiple integral]
Let $f(\mathbf x)$ be a real-valued function defind 
on a boxed region 
	\[V = [a_0^0, a_1^0]\times [a_0^1,a_1^1]\times \cdots\times[a_0^{n-1}, a_1^{n-1}].\]
We say $f$ is \textbf{integrable} if the limit
	\[\lim_{N\to\infty}\sum_{j_0,\ldots,j_{n-1}=1}^{N} f(\mathbf x_j^*)
	\Delta x^0\Delta x^{n-1}\]
exists for arbitrary choice of $\mathbf x_j^* \in 
\displaystyle\prod_{i=0}^{n-1}[x_{j-1}^{i}, x_{j}^{i}]$
where $x_j^i = (a_1^i - a_0^i)/N$.
The limit is called the \textbf{multiple integral} of $f$ on $V$.
\end{defn}
\end{frame}

\begin{frame}
The multiple integral of $f$ on $3$-dimensional boxed region $V$ is 
called the \emph{triple integral}:
	\[\iiint_Vf(x,y,z)dxdydz =\mkern-20mu 
	\iiint\displaylimits_{[a,b]\times[c,d]\times[e,f]}\mkern-18mu F(x,y,z)dxdydz\]
\begin{defn}
Let $f(x,y,z)$ be a function defined on a bounded region $V\subset\mathbf R^3$
which is contained in a boxed region $[a,b]\times[c,d]\times[e,f]$.
We say $f$ is integrable if the function $F(x,y,z)$ defined on 
$[a,b]\times[c,d]\times[e,f]$ as below is integrable:
	\[F(x,y,z) = \begin{dcases}
		f(x,y,z) & (x,y,z) \in V \\
		0 & \textrm{otherwise.}
	\end{dcases}\]
\end{defn}
% \begin{rem}
% Let $f(x,y,z)$ be a function defined on the boxed region 
% $[a,b]\times[c,d]\times[e,f]\subset\mathbf R^3$.
% The integral of $f$ over $[a,b]\times[c,d]\times[e,f]$ 
% is called the \textbf{triple integral} of $f$ and denoted by
% \[\iiint\displaylimits_{[a,b]\times[c,d]\times[e,f]}\mkern-18mu f(x,y,z)dxdydz\]
% \end{rem}
\end{frame}

\begin{frame}
\begin{thm}
Let $f(x,y,z)$ be a continuous function defined on 
a bounded region $V\subset\mathbf R^3$.
Then $f$ is integrable.
\end{thm}
\begin{thm}[Fubini III]
Let $f(x,y,z)$ be a continuous function defined on the region $V$.
	\[V=\{(x,y,z)\,|\,a\le x\le b,\,g_1(x)\le y\le g_2(x),\, h_1(x,y)\le z\le h_2(x,y)\}\]
Then the following holds.
	\[\iiint_Vf(x,y,z)dxdydz = \int_a^b \int_{g_1(x)}^{g_2(x)} \int_{h_1(x,y)}^{h_2(x,y)}f(x,y,z)dz dy dx\]
\end{thm}
\end{frame}

\begin{frame}
\begin{figure}[h]
\begin{center}
\includegraphics[scale=.4]{image/lnsec9-2}
\end{center}
\end{figure}
\end{frame}

\begin{frame}
\begin{exmp}
Let $V$ be a parallelopiped region bounded by $6$ planes : $2x=y$, $2x=y+2$, $y=0$, $y=4$, $z=0$, $z=3$.
Compute
	\[\iiint_V\frac{2x-y}{2}+\frac{z}{3}dxdydz\]
\end{exmp}
\end{frame}


\begin{frame}
\begin{prob}
For $f(x,y) = x^2-y^2$, compute $\displaystyle \iint_S x+z dS$.
\end{prob}
\end{frame}

\begin{frame}
\begin{prob}
Let $f(x,y)$ is a density at the point $(x,y)$ on domain $D$.
Let
	\[\bar x= \frac{\displaystyle\iint_D xf(x,y)dA}{\displaystyle\iint_Df(x,y)dA},\quad 
	\bar y = \frac{\displaystyle\iint_D yf(x,y)dA}{\displaystyle\iint_Df(x,y)dA}.\]
Explain why $(\bar x,\bar y)$ is the center of mass of $D$.
\end{prob}
\end{frame}

\begin{frame}
\begin{prob}
Let $V=\{0\le x\le 2,0\le y\le x,0\le z\le y\}$.
Write $displaystyle\iiint_V f(x,y,z)$ in six different iterated integrals.
\end{prob}
\end{frame}

\end{document}