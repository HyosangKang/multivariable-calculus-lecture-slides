\documentclass[t]{beamer}
\usefonttheme{serif}

\usepackage{amsmath,amsthm,amssymb,amsfonts,amscd,mathrsfs,amsxtra,multirow,kotex,mathtools,gensymb,textcomp,lipsum,tikz,verbatim,color,soul,courier,mdframed,xcolor}
\usepackage[normalem]{ulem}
\usetikzlibrary{calc,matrix,arrows,chains,positioning,scopes}
\usepackage{pdfpages}

\theoremstyle{plain}
\newtheorem{thm}{Theorem}[section]
\newtheorem{prop}[thm]{Proposition}

\theoremstyle{definition}
\newtheorem{defn}[thm]{Definition}
\newtheorem{exmp}[thm]{Example}
\newtheorem{excs}[thm]{Exercise}
\newtheorem{rem}[thm]{Remark}
\newtheorem{prob}[thm]{Problem}
\newtheorem{cor}[thm]{Corollary}

\newcommand \tr[1]{\textcolor{red}{#1}}
\newcommand{\tikzmark}[1]{\tikz[overlay,remember picture] \node (#1) {};}
\newcommand{\varep}{\varepsilon}
\newcommand{\DrawBox}[1][]{%
    \tikz[overlay,remember picture]{
    \draw[red,#1]
      ($(left)+(-0.2em,0.9em)$) rectangle
      ($(right)+(0.2em,-0.3em)$);}
}

\newcommand{\tikzmarkk}[2]{
    \tikz[overlay,remember picture,baseline] 
    \node[anchor=base] (#1) {$#2$};
}
\newcommand*\circled[1]{\tikz[baseline=(char.base)]{
            \node[shape=circle,draw,inner sep=2pt] (char) {#1};}}

\tikzset{join/.code=\tikzset{after node path={%
\ifx\tikzchainprevious\pgfutil@empty\else(\tikzchainprevious)%
edge[every join]#1(\tikzchaincurrent)\fi}}}

\tikzset{>=stealth',every on chain/.append style={join},
         every join/.style={->}}
\tikzstyle{labeled}=[execute at begin node=$\scriptstyle,
   execute at end node=$]

\newenvironment<>{proofs}[1][\proofname]{%
   \par
   \def\insertproofname{#1\@{.}}%
   \usebeamertemplate{proof begin}#2}
 {\usebeamertemplate{proof end}}
 

\addtobeamertemplate{navigation symbols}{}{%
    \usebeamerfont{footline}%
    \usebeamercolor[fg]{footline}%
    \hspace{1em}%
    \raisebox{2pt}[0pt][0pt]{\insertframenumber/\inserttotalframenumber}
}
\setbeamercolor{footline}{fg=blue}
\setbeamerfont{footline}{series=\bfseries}
\title[]{SE102:Multivariable Calculus}

\author[]{Hyosang Kang\inst{1}}

\institute[]{\inst{1}Division of Mathematics\\ School of Interdisciplinary Studies\\ DGIST}

\date[]{Week 07}

\begin{document}

\begin{frame}
\titlepage
\end{frame}

\begin{frame}
\begin{exmp}
Find all extremals of $f(x,y)=xy-y+x-2$
on the region $x^2+y^2\le2$.
\end{exmp}
\end{frame}

\begin{frame}
\begin{exmp}
Find all extremals of 
$f(x,y) = - x^2 + 3xy - 2y^2$ on the region 
$2x^2 - 6xy + 5y^2\le 1$.
\end{exmp}
\end{frame}

\begin{frame}
\begin{exmp}
Find the dimensions of the cube inscribed in the
sphere of radius $2$ whose surface area is maximum.
\end{exmp}
\end{frame}

\begin{frame}
\begin{exmp}
Find the point on the surface $x^3+y^2+z=2$ 
closest to the origin.
\end{exmp}
\end{frame}
    
\begin{frame}
\begin{exmp}
Suppose  $f(x,y)$ is a differentiable function defined 
on $\mathbf R^2$. If $(x_0,y_0)$ is a critical point, 
explain why it is a saddle point if the Hessian 
$H_f(x_0,y_0)$ is negative regardless of the value of 
$f_{xx}(x_0,y_0)$. 
\end{exmp}
\end{frame}

\begin{frame}
\begin{exmp}
Let $\mathbf v$, $\mathbf w$, $\mathbf u$ be
linearly independent $3$-dimensional position vectors.
Show that the volume of the parallelogram bounded by 
these vectors is 
$\left|(\mathbf v\times\mathbf w)\cdot\mathbf u\right|$.
\end{exmp}
\end{frame}

\begin{frame}[t]
\begin{exmp}
Discuss the difference on geometric configurations
of three vectors $\mathbf v$, $\mathbf w$, $\mathbf u$
when the value of
$(\mathbf v\times\mathbf w)\cdot\mathbf u$ is 
positive, negative, or zero.
\end{exmp}
\end{frame}

\begin{frame}[t]
\begin{exmp}
Prove or disprove: 
for any vectors $\mathbf u$, $\mathbf v$, $\mathbf w$,
\begin{enumerate}
\item $\mathbf u\cdot(\mathbf v\times\mathbf w)
 = (\mathbf u\times\mathbf w)\cdot\mathbf w$.
\item $\mathbf u\times(\mathbf v\times\mathbf w)
 = (\mathbf u\times\mathbf v)\times\mathbf w$.
\end{enumerate}
\end{exmp}
\end{frame}

\begin{frame}
\begin{exmp}
Determine whether the limit exists:
$$\lim_{(x,y)\rightarrow(0,0)}
\frac{xy+yx^2}{x^2+y^2}$$ 
\end{exmp}
\end{frame}

\begin{frame}
\begin{exmp}
Determine whether the limit exists:
$\displaystyle \lim_{(x,y)\rightarrow(0,0)}
\frac{x^2\sin^2y}{x^2+2y^2}$
\end{exmp}
\end{frame}

\begin{frame}
\begin{exmp}
Determine whether the function is continuous at $(0,0)$.
$$f(x,y) = \begin{dcases}
\frac{xy}{x^2+xy+y^2} & (x,y)\neq(0,0)\\
0 & (x,y) = (0,0)\end{dcases}$$
\end{exmp}
\end{frame}

\begin{frame}
\begin{exmp}
Find all points where the directional derivative
of $f(x,y)=x^2+y^2-2x-4y$ to the vector 
$\mathbf u = \frac{1}{2} (1,1)$ is maximized.
\end{exmp}
\end{frame}

\begin{frame}
\begin{exmp}
Given a fixed $c>0$, 
show that the sum of three intercepts of any tangent plane
to the surface $\sqrt x+\sqrt y+\sqrt z=\sqrt c$
is constant.
\end{exmp}    
\end{frame}

\begin{frame}
\begin{exmp}
Consider a small circle of radius $b$
rolling inside the larger circle of radius $a$ ($a>b$).
Find the parametric equations of 
the trajectory of the point on the small circle.
\end{exmp}
\end{frame}
\end{document}