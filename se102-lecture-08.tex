\documentclass[t]{beamer}
\usefonttheme{serif}

\usepackage{amsmath,amsthm,amssymb,amsfonts,amscd,mathrsfs,amsxtra,multirow,kotex,mathtools,gensymb,textcomp,lipsum,tikz,verbatim,color,soul,courier,mdframed,xcolor}
\usepackage[normalem]{ulem}
\usetikzlibrary{calc,matrix,arrows,chains,positioning,scopes}
\usepackage{pdfpages,cancel}

\theoremstyle{plain}
\newtheorem{thm}{Theorem}[section]
\newtheorem{prop}[thm]{Proposition}

\theoremstyle{definition}
\newtheorem{defn}[thm]{Definition}
\newtheorem{exmp}[thm]{Example}
\newtheorem{excs}[thm]{Exercise}
\newtheorem{rem}[thm]{Remark}
\newtheorem{prob}[thm]{Problem}
\newtheorem{cor}[thm]{Corollary}

\newcommand \tr[1]{\textcolor{red}{#1}}
\newcommand{\tikzmark}[1]{\tikz[overlay,remember picture] \node (#1) {};}
\newcommand{\varep}{\varepsilon}
\newcommand{\DrawBox}[1][]{%
    \tikz[overlay,remember picture]{
    \draw[red,#1]
      ($(left)+(-0.2em,0.9em)$) rectangle
      ($(right)+(0.2em,-0.3em)$);}
}

\newcommand{\tikzmarkk}[2]{
    \tikz[overlay,remember picture,baseline] 
    \node[anchor=base] (#1) {$#2$};
}
\newcommand*\circled[1]{\tikz[baseline=(char.base)]{
            \node[shape=circle,draw,inner sep=2pt] (char) {#1};}}

\tikzset{join/.code=\tikzset{after node path={%
\ifx\tikzchainprevious\pgfutil@empty\else(\tikzchainprevious)%
edge[every join]#1(\tikzchaincurrent)\fi}}}

\tikzset{>=stealth',every on chain/.append style={join},
         every join/.style={->}}
\tikzstyle{labeled}=[execute at begin node=$\scriptstyle,
   execute at end node=$]

\newenvironment<>{proofs}[1][\proofname]{%
   \par
   \def\insertproofname{#1\@{.}}%
   \usebeamertemplate{proof begin}#2}
 {\usebeamertemplate{proof end}}
 

\addtobeamertemplate{navigation symbols}{}{%
    \usebeamerfont{footline}%
    \usebeamercolor[fg]{footline}%
    \hspace{1em}%
    \raisebox{2pt}[0pt][0pt]{\insertframenumber/\inserttotalframenumber}
}
\setbeamercolor{footline}{fg=blue}
\setbeamerfont{footline}{series=\bfseries}
\title[]{SE102:Multivariable Calculus}

\author[]{Hyosang Kang\inst{1}}

\institute[]{\inst{1}Division of Mathematics\\ School of Interdisciplinary Studies\\ DGIST}

\date[]{Lecture 08\\
Manifolds}

\begin{document}

\begin{frame}
\titlepage
\end{frame}

\begin{frame}
\begin{defn}
The integral $\displaystyle\iint_Df(x,y)dxdy$ is called \textbf{improper} if it satisfies one of the following.
\begin{itemize}
\item the region $D$ is unbounded, or
\item the function diverges at some point in $D$.
\end{itemize}
\end{defn}
\end{frame}

\begin{frame}
\begin{exmp}
Let $D= \mathbf R^2$ be the entire $2$-dimensional plane. 
Let us compute 
	\[\iint_{\mathbf R^2}e^{-x^2-y^2}dxdy\]
By polar coordinate $T(r,\theta)=(r\cos\theta,r\sin\theta)$,
	\[T^{-1}(D) = [0,\infty)\times[0,2\pi].\]	
Thus by change of coordinates,
	$$\iint_{\mathbf R^2}e^{-x^2-y^2}dxdy = \int_0^\infty \int_0^{2\pi} e^{-r^2}rd\theta dr 
	= 2\pi \left(\frac{-1}{2}e^{-r^2}\right)\Big|_0^\infty = \pi$$
\end{exmp}
\end{frame}

\begin{frame}
\begin{defn}\label{defn-Gamma}
The \textbf{gamma function} $\Gamma:\mathbf R\rightarrow \mathbf R$ is defined by
	\[\Gamma(p)=\int_0^\infty x^{p-1}e^{-x}dx\]
\end{defn}
\begin{exmp}
\begin{enumerate}
	\item $\Gamma(n)=(n-1)!$ for $n\ge1$.
	\item $\Gamma(x)$ diverges at each non-positive integer $x$.
	\item $\Gamma\left(\frac{1}{2}\right)=\sqrt\pi$.
\end{enumerate}
\end{exmp}
\end{frame}

\begin{frame}
\begin{defn}
The \textbf{beta function} $B(x,y):\mathbf R^2\to\mathbf R$ is defined by
	\[B(x,y) = \int_0^1t^{x-1}(1-t)^{y-1}dt\]
\end{defn}

\begin{prop}
	\begin{enumerate}
		\item $\displaystyle B(x,y)=B(y,x)$
		\item $\displaystyle B(x,y) = \frac{\Gamma(x)\Gamma(y)}{\Gamma(x+y)}$
	\end{enumerate}
\end{prop}
\end{frame}

\begin{frame}
\begin{defn}
The $n$-dimensional ball of radius $r$
is the set of points in $4$-dimensional space 
defined by 
	\[B_n(r) = \{(x_1,\cdots,x_n)\mid x_1^2+\cdots+x_n^2\le r^2\}.\]
The $n-1$-dimensional sphere of radius $r$
is the boundary of $B_n(r)$, defined by 
	\[S_{n-1}(r) = \{(x_1,\cdots,x_n)\mid x_1^2+\cdots+x_n^2=1\}.\]
\end{defn}
\end{frame}

\begin{frame}
\begin{prop}
Let $T:\mathbf R^n\to\mathbf R^n$ be a 
transformation (i.e. one-to-one, differentiable)
such that 
	\[T(u_1,\cdots,u_n) = (x_1,\cdots,x_n).\]
If $U$ be a region in $\mathbf R^n$ and $V=T(U)$.
Then 
	$$\int_Vdx_1\cdots dx_n
	= \int_U\left|\det\frac{\partial(x_1,\cdots,x_n)}{\partial(u_1,\cdots,u_n)}\right|
	du_1\cdots du_n$$
\end{prop}
\end{frame}

\begin{frame}
We define the \textbf{volume} of the cube $[0,1]^n$ to be $1$.
There are 3 ways to compute the volume of $4$-dimensional ball.
\begin{enumerate}
	\item Using spherical coordinate.
	\item Integrating sections.
	\item Finding recursive formula.
\end{enumerate}
\end{frame}

\begin{frame}
\begin{exmp}
Let $T:[0,1]\times[0,\pi]\times[0,\pi]\times[0,2\pi]\to\mathbf R^4$
be a transformation defined by
	\begin{align*}
	T(r,\theta_1,\theta_2,\phi)
	&= (r\sin\theta_1\sin\theta_2\cos\phi,
	r\sin\theta_1\sin\theta_2\sin\phi,\\
	&r\sin\theta_1\cos\theta_2,
	r\cos\theta_1)
	\end{align*}
Such $T$ is called a $4$-spherical transformation
and the Jacobian is
	\[J_T = r^3\sin^2\theta_1\sin\theta_2\]
Thus the volume of $B_4(1)$ is
	$$\int_0^1\int_0^\pi\int_0^\pi\int_0^{2\pi}
	r^3\sin^2\theta_1\sin\theta_2d\phi d\theta_2
	d\theta_1dr = \frac{\pi^2}{2}$$
\end{exmp}
\end{frame}

\begin{frame}
\begin{exmp}
As we slice the $4$-dimensional ball $B_4(1)$
at each $w$-coordinate, we obtain $3$-dimensional
ball of radius $\sqrt{1-w^2}$.
Thus the volume of $B_4(1)$ is
	\[\int_{-1}^1\textrm{vol} B_3(\sqrt{1-w^2})dV dw\]
Since we know $\displaystyle \textrm{vol}B_3(r) = \frac{4\pi}{3}r^3$,
we can compute the integral using Gamma and Beta functions.
\end{exmp}
\end{frame}

\begin{frame}
\begin{exmp}	
The ball $B_4(1)$ is the union of $3$-dimensional
spheres $S_3(r)$ for $0\le r\le 1$.
Thus the volume of $B_4(1)$ is
	\[\int_0^1 S_3(r)dr = \int_0^1\textrm{vol}S_3(1)r^3dr = \textrm{vol}S_3(1)/4\]
Meanwhile, the $3$-dimensional sphere $S_3(1)$ is the union of
product of two circles $S_1(r)\times S_1(r')$ where $r^2+r'^2=1$.
Thus the volume of $S_3(1)$ is
	\[\int_0^{\pi/2}\textrm{vol}S_1(r)\textrm{vol}S_1(r')d\theta=2\pi^2\]
\end{exmp}
\end{frame}

\begin{frame}
Multivariable Calculus summerizes in two sentences:
\begin{itemize}
	\item Derivative is a linear transformation.
		\begin{itemize}
			\item Derivative $Df(\mathbf a)$ of a multivariable function
			is a linear map between tangent spaces at $\mathbf a$ and $f(\mathbf a)$.
		\end{itemize}
	\item Divergence theorem is a Stokes' theorem.
		\begin{itemize}
			\item The general form of Stokes' theorem is
				\[\int_V d\omega = \int_{\partial V}\omega\]
			where $\omega$ is a differential $(k-1)$-form 
			and $V$ is a $k$-dimensional space. 
			The differential $d\omega$ is a $k$-form.
			The Stokes theorem is when 
				\[\omega = Pdx + Qdy = Rdz\]
			and divergence theorem is when 
				\[\omega = Pdx\wedge dy + Qdy\wedge dz + Rdz\wedge dx.\]
		\end{itemize}
\end{itemize}
\end{frame}

\begin{frame}
\begin{defn}
Let $\mathcal C^\infty(\mathbf R^n)$ be the set of all $\mathcal C^\infty$-class real-valued functions 
defined on $\mathbf R^n$. A tangent vector $\mathbf v$ at $\mathbf p\in\mathbf R^n$ is a
map $\mathcal C^\infty(\mathbf R^n)\to\mathbf R$ such that
there is a parametric curve $\mathbf c(t)$ of $\mathcal C^\infty$-class that satisfies
\begin{itemize}
	\item $\mathbf c(0)=\mathbf p$, and
	\item $\mathbf v(f) = (f\circ\mathbf c)'(0)$.
\end{itemize}
\end{defn}
\begin{rem}
A tangent vector at $\mathbf p$ in $\mathbf R^n$ is usually denoted as 
	\[\mathbf v = a_0\left.\frac{\partial}{\partial x^0}\right|_{\mathbf p} 
	+ \cdots + a_{n-1}\left.\frac{\partial}{\partial x^{n-1}}\right|_{\mathbf p}\]	
for real coefficients $a_0,\ldots,a_{n-1}$.
\end{rem}
\end{frame}

\begin{frame}
\begin{defn}
A $n$-dimensional \textbf{vector field} $\mathbf F$ is a map assigns a tangent vector at $\mathbf p$ 
to each point $\mathbf p$ in $\mathbf R^n$.
We can write $\mathbf F$ as 
	\[\mathbf F(\mathbf p) = a_0(\mathbf p)\frac{\partial}{\partial x^0} 
	+ \cdots + a_{n-1}(\mathbf p)\frac{\partial}{\partial x^{n-1}}\]	
for real-valued function $\mathbf a_i$.	
We say $\mathbf F$ is of class $\mathcal C^\infty$ if all coefficient functions $a_i$ are of class $\mathcal C^\infty$.
\end{defn}
\begin{rem}
The set of all tangent vectors at $\mathbf p$ is denoted by $T_{\mathbf p}\mathbf R^n$,
and it is a $n$-dimensional vector space.
A $\mathcal C^\infty$-vector field $\mathbf F$ is a map that sends $\mathcal C^\infty$-functions
to $\mathcal C^\infty$-functions.
The set of all $\mathcal C^\infty$-vector fields on $\mathbf R^n$ is denoted by $\mathfrak{X}(\mathbf R^n)$.
As a vector space $\mathfrak X(\mathbf R^n)$ is generated by $\displaystyle\frac{\partial}{\partial x^j}$, $j=0,\ldots, n-1$.
\end{rem}
\end{frame}

\begin{frame}
\begin{defn}
Let $V$ be a $n$-dimensional vector space.
The \textbf{dual space} $V^*$ of $V$ is the vector space defined by 
	\[ V^* = \{f:V\to\mathbf R\,|\, f\textrm{ is linear}\}.\]
Let $e_0,\ldots,e_{n-1}$ be a basis for $V$. 
The linear map $e^*_i:V\to\mathbf R$ defined by 
	\[e_i^*(e_j) = \begin{dcases}
		0 & i\neq j \\
		1 & i =j 
	\end{dcases}\]
is called a dual vector to $e_i$. 
\end{defn}
\begin{defn}
Let $dx^j$ be the dual vector to the vector field $\displaystyle\frac{\partial}{\partial x^j}$.
The $n$-dimensional $1$-\textbf{form} $\omega$ is a linear combination of $dx^j$ 
with coefficients in $\mathcal C^\infty(\mathbf R^n)$.
	\[\omega = f_0dx^0 + \cdots + f_ndx^{n-1}.\]
\end{defn}
\end{frame}

\begin{frame}
\begin{defn}
The \textbf{wedge} product between two $n$-dimensional $1$-forms $dx^i$ and $dx^j$,
denoted by $dx^i\wedge dx^j$ satisfies 
\begin{itemize}
	\item $dx^i\wedge dx^j = -dx^j\wedge dx^i$;
	\item $dx^i\wedge dx^i = 0$.
\end{itemize}
A \textbf{$k$-form} is a linear combination of wedge products on $k$ $1$-forms. 
The $n$-dimensional $n$-form is called the \textbf{volume form}. 
\end{defn}
\end{frame}

\begin{frame}
\begin{rem}
The integration of $n$-form $dx_1\wedge\cdots\wedge dx_n$ on the cube $[0,1]^n$ is defined by 
	\[\int_{[0,1]^n}dx_1\wedge\cdots\wedge dx_n = 1.\]
The volume of $n$-dimensional region $V\subset\mathbf R^n$ is defined by 
	\[\int_V dx_1\wedge\cdots\wedge dx_n.\]
Anti-commutivity of the wedge product ($dx\wedge dy = -dy \wedge dx$) implies
that the integration of $n$-form is a generalization of surface (and line) integrals of vector fields.
\end{rem}
\end{frame}

\begin{frame}
\begin{defn}
Let $\mathcal T^k(\mathbf R^n)$ be the set of all $k$-forms on $\mathbf R^n$.
The $d$-operator is a linear map $d:\mathcal T^k(\mathbf R^n)\to\mathcal T^{k+1}(\mathbf R^n)$ defined by 
	\[d(fdx^{i_1}\wedge \cdots \wedge dx^{i_k}) 
	= \sum_{j=0}^{n-1}\frac{\partial f}{\partial x^j}dx^j\wedge dx^{i_1}\wedge \cdots\wedge d^{i_k}.\]
\end{defn}
\begin{thm}[Stokes]
Let $\omega$ be a $k-1$-form defined a bounded region $V\subset\mathbf R^n$. Then
	\[\int_V d\omega = \int_{\partial V} \omega.\]
\end{thm}
\end{frame}

\begin{frame}
\begin{prob}
	Compute the improper integral $\displaystyle\int_{-\infty}^\infty e^{-x^2}dx$.
\end{prob}
\end{frame}

\begin{frame}
\begin{prob}
	Find the volume of $B_5(1)$ using
	\begin{enumerate}
		\item spherical coordinates on $5$-dimensional space.
		\item Gamma and Beta functions.
		\item recursive formula on volunme of $n$-dimensional balls.
	\end{enumerate}
\end{prob}
\end{frame}

\end{document}
