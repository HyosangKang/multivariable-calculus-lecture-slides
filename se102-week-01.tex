\documentclass{beamer}
\usefonttheme{serif}

\usepackage{amsmath,amsthm,amssymb,amsfonts,amscd,mathrsfs,amsxtra,multirow,kotex,mathtools,gensymb,textcomp,lipsum,tikz,verbatim,color,soul,courier,mdframed,xcolor}
\usepackage[normalem]{ulem}
\usetikzlibrary{calc,matrix,arrows,chains,positioning,scopes}
\usepackage{pdfpages}

\theoremstyle{plain}
\newtheorem{thm}{Theorem}[section]
\newtheorem{prop}[thm]{Proposition}

\theoremstyle{definition}
\newtheorem{defn}[thm]{Definition}
\newtheorem{exmp}[thm]{Example}
\newtheorem{excs}[thm]{Exercise}
\newtheorem{rem}[thm]{Remark}
\newtheorem{prob}[thm]{Problem}

\newcommand \tr[1]{\textcolor{red}{#1}}
\newcommand{\tikzmark}[1]{\tikz[overlay,remember picture] \node (#1) {};}
\newcommand{\varep}{\varepsilon}
\newcommand{\DrawBox}[1][]{%
    \tikz[overlay,remember picture]{
    \draw[red,#1]
      ($(left)+(-0.2em,0.9em)$) rectangle
      ($(right)+(0.2em,-0.3em)$);}
}

\newcommand{\tikzmarkk}[2]{
    \tikz[overlay,remember picture,baseline] 
    \node[anchor=base] (#1) {$#2$};
}
\newcommand*\circled[1]{\tikz[baseline=(char.base)]{
            \node[shape=circle,draw,inner sep=2pt] (char) {#1};}}

\tikzset{join/.code=\tikzset{after node path={%
\ifx\tikzchainprevious\pgfutil@empty\else(\tikzchainprevious)%
edge[every join]#1(\tikzchaincurrent)\fi}}}

\tikzset{>=stealth',every on chain/.append style={join},
         every join/.style={->}}
\tikzstyle{labeled}=[execute at begin node=$\scriptstyle,
   execute at end node=$]


\addtobeamertemplate{navigation symbols}{}{%
    \usebeamerfont{footline}%
    \usebeamercolor[fg]{footline}%
    \hspace{1em}%
    \raisebox{2pt}[0pt][0pt]{\insertframenumber/\inserttotalframenumber}
}
\setbeamercolor{footline}{fg=blue}
\setbeamerfont{footline}{series=\bfseries}
\title[]{SE102:Multivariable Calculus}

\author[]{Hyosang Kang\inst{1}}

\institute[]{\inst{1}Division of Mathematics\\ School of Interdisciplinary Studies\\ DGIST}

\date[]{Week 01}

\begin{document}

\begin{frame}
\titlepage
\end{frame}

\begin{frame}
\begin{defn}\label{defn-vec}
A ($n$-dimensional) \textbf{vector} is a $n$-tuple of real numbers 
	\[\mathbf a=(a_1,a_2,\cdots,a_n)\]
with two operations.
	\begin{itemize}
	\item (vector sum) For a vector $\mathbf b = (b_1,b_2,\cdots,b_n)$,
		\[\mathbf a+\mathbf b=(a_1+b_1,a_2+b_2,\cdots,a_n+b_n).\]
	\item (scalar multiplication) For $k\in\mathbb R$, we define a vector $k\mathbf a$ as
		\[k\mathbf a=(ka_1,ka_2,\cdots,ka_n).\]
	\end{itemize}

The ($n$-dimensional) \textbf{vector space} is the set 
of all ($n$-dimensional) vectors, and we denote it as $\mathbb R^n$.
\end{defn}
\end{frame}

\begin{frame}
\begin{exmp}
Let $O=(0,0,0)$ be the origin and $P=(a_1,a_2,a_3)$ a point in $3$-dimensional space.
A \textbf{position vector} from $O$ to $P$ is a vector 
	\[\overrightarrow{OP}=(a_1,a_2,a_3).\]
The scalar multiplication of a position vector is a dilation.
The vector sum of two position vectors is a superposition.

Let $Q=(b_1,b_2,b_3)$ and $R=(a_1+b_1,a_2+b_2,a_3+b_3)$.
the vector $\overrightarrow{PR}$ is the same as $\overrightarrow{OQ}$.
We have the following additive operation.
	\[\overrightarrow{OP}+\overrightarrow{OQ}=\overrightarrow{OP}+\overrightarrow{PR}=\overrightarrow{OR}\]
\end{exmp}
\end{frame}

\begin{frame}
\begin{exmp}
A parametrization of a curve is a set of functions defined on a common interval,
each indicates a coordinate of a point on the curve.
Let $c:(-\varep,\varep)\rightarrow\mathbb R^2$ be a parametrization of a curve on a plane.
	\[c(t)=(x(t),y(t))\]
The differntial $c'(t)=(x'(t),y'(t))$ is a vector which represents 
the \textbf{velocity} of the parametrization at $t$. 
The $x,y$-components of the vector $c'(t)$ represent 
the projection of the speed of $c(t)$ in $x,y$-directions respectively.
We can decompose the velocity vector into the sum of horizontal and vertical velocity of $c(t)$.
	\[c'(t) = (x'(t),0)+(0,y'(t))\]
\end{exmp}
\end{frame}

\begin{frame}
\begin{exmp}
Let $\mathbb R^n$ be the set of all $n$-dimensional space.
Let $f:\mathbb R^2\to\mathbb R^3$ be a function defined as
	\begin{equation}\label{eqn-mult-exmp-surf}
	f(u,v)=(x(u,v),y(u,v),z(u,v)).
	\end{equation}
The variables $x,y,z$ are dependent to the variables $u,v$.
The functions $f$ (and also $x,y,z$ as functions) is called a 
\textbf{multivariable function}
since it contains two or more independent or dependent variables.
If we write $\mathbf a=(u,v)$ and $\mathbf b=(x,y,z)$,
the equation \eqref{eqn-mult-exmp-surf} can be written as $f(\mathbf a)=\mathbf b$.
\end{exmp}
\end{frame}

\begin{frame}
\begin{exmp}
Let us parametrize a line $l$ in $\mathbf R^3$.
\begin{enumerate}
	\item Suppose that $l$ passes through $P=(x_0,y_0,z_0)$ and parallel to $\mathbf a=(a_1,a_2,a_3)$. Then the paramtric equations of $l$ is
		\begin{equation}\label{eqn-param-lin}
		l(t) = P + t\mathbf a
		\end{equation}
	Equivalently, equation \eqref{eqn-param-lin} can be written as a \emph{symmetric form} as follows.
		\begin{equation}\label{eqn-param-lin2}
		\frac{x-x_0}{a_1} = \frac{y-y_0}{a_2} = \frac{z-z_0}{a_3}
		\end{equation}
    \item Suppoe that $l$ passes through two points $P,Q$. By subsitute $\mathbf a=\overrightarrow{OP}-\overrightarrow{OQ}$ to \eqref{eqn-param-lin} or \eqref{eqn-param-lin2}, we get a parametric equation of $l$.
\end{enumerate}
\end{exmp}
\end{frame}


\begin{frame}
\begin{defn}
Let $\mathbf a=(a_1,a_2,\cdots,a_n)$ be a ($n$-dimensional) vector. The \textbf{norm}
of $\mathbf a$ is the value
	\[\Vert \mathbf a\Vert =\sqrt{a_1^2+a_2^2+\cdots+a_n^2}.\]
We call a vector $\mathbf a$ a unit vector if $\Vert a\Vert = 1$.
The \textbf{zero vector} $\mathbf 0$ is the vector satisfying 
	\[\mathbf 0+\mathbf a = \mathbf a + \mathbf 0 = \mathbf a.\]
for any vector $\mathbf a$.
Note that $\Vert \mathbf a\Vert=0$ if and only if $\mathbf a=\mathbf 0$.
The \textbf{normalization} of $\mathbf a$ is the vector
	\[\mathbf u=\mathbf a/\Vert \mathbf a\Vert \]
\end{defn}
\end{frame}

\begin{frame}
\begin{exmp}
For each $i=1,\cdots, n$, 
the (unit) \textbf{basis vector} is
	\[\mathbf e_i=(0,\cdots,1,\cdots,0).\]
($1$ is at the $i$ th place.)
In particular, we denote $3$-dimensional basis vectors as
	$$\mathbf i=(1,0,0),
		\quad\mathbf j=(0,1,0),
		\quad\mathbf k=(0,0,1)$$
We can decompose any vector 
$\mathbf a=(a_1,a_2,\cdots,a_n)$ as a linear sum of 
basis vectors:
	$$\mathbf a 
	= a_1\mathbf e_1+a_2\mathbf e_2+\cdots+a_n\mathbf e_n$$
Thus, we call the set
$\{\mathbf e_1,\mathbf e_2,\cdots,\mathbf e_n\}$
as a \textbf{basis} of $\mathbf R^n$.
\end{exmp}
\end{frame}

\begin{frame}
\begin{defn}
The \textbf{inner product} (also called \textbf{dot product}) of two vectors $\mathbf a$, $\mathbf b$ is an operation defined by
	\[\mathbf a\cdot\mathbf b=a_1b_1+a_2b_2+\cdots+a_nb_n\]
If $\mathbf a\cdot\mathbf b=0$, then we say $\mathbf a$, $\mathbf b$ are \textbf{orthogonal}.
\end{defn}
\begin{prop}
The inner product satisfies the following.
\begin{enumerate}
\item $\mathbf a\cdot(\mathbf b+\mathbf c)=\mathbf a\cdot\mathbf b+\mathbf a\cdot\mathbf c$
\item $\mathbf a\cdot(k\mathbf b)=k(\mathbf a\cdot\mathbf b)$
\item $\mathbf a\cdot\mathbf a=\Vert \mathbf a\Vert ^2$
\end{enumerate}
\end{prop}
\end{frame}



\begin{frame}
\begin{thm}\label{thm-dot-cos}
Let $\mathbf a$, $\mathbf b$ be nonzero $2$-dimensional vectors.
If $\theta$ is the angle between $\mathbf a$ and $\mathbf b$, then 
	\[\mathbf a\cdot\mathbf b=\Vert \mathbf a\Vert \cdot\Vert \mathbf b\Vert \cdot\cos\theta\]
\end{thm}
\begin{defn}
For nonzero vectors $\mathbf a$ and $\mathbf b$ with the same dimension, 
the vector defined by
	\[\textup{proj}_{\mathbf b}\mathbf a=\left(\frac{\mathbf a\cdot\mathbf b}{\Vert \mathbf b\Vert ^2}\right)\mathbf b\]
is called the \textbf{projection of $\mathbf a$ onto $\mathbf b$}.
\end{defn}
\end{frame}

\begin{frame}
\begin{rem}
As norm measures the \emph{size} of a vector, the inner product measure the \emph{direction}.
For example, the direction of a $2$-dimensional vector $\mathbf a$ is determined by $0\le \theta_1,\theta_2\le\pi$ satisfying
	\[\mathbf a\cdot\mathbf e_1=\Vert \mathbf a\Vert \cos\theta_1\]
	\[\mathbf a\cdot\mathbf e_2=\Vert \mathbf a\Vert \cos\theta_2\]
In order to determine the direction of a $3$-dimensional vector, say $\mathbf a$, 
we need three angles $0\le\alpha,\beta,\gamma\le\pi$ satisfying
	\begin{equation}\label{eqn-direc-cosi}
	\cos\alpha=\frac{\mathbf a\cdot\mathbf i}{\Vert \mathbf a\Vert },\quad \cos\beta=\frac{\mathbf a\cdot\mathbf j}{\Vert \mathbf a\Vert },
	\quad\cos\gamma=\frac{\mathbf a\cdot\mathbf k}{\Vert \mathbf a\Vert }
	\end{equation}
Such quantities are called the \textbf{direction cosines}. 
\end{rem}
\end{frame}

\begin{frame}
\begin{exmp}
Let us parametrize a plane $P$ in $\mathbf R^3$.
\begin{enumerate}
	\item Suppose that the plane $P$ contains a point $A  = (x_0,y_0,z_0)$ and the vector $\mathbf n$ is normal to $P$. Then for any point $X=(x,y,z)$, the vector $\overrightarrow{AX}$ is perpendicular to the vector $\mathbf n$. This property can be written as $\mathbf n\cdot\overrightarrow{AX}=0$ and more specifically, if $\mathbf n = (a,b,c)$, 
	\[a(x-x_0)+b(y-y_0)+c(z-z_0)=0\]
    \item Suppose that the plane $P$ contains two vectors $\mathbf a,\mathbf b$ which are not parallel and passes through a point $\mathbf x_0$. Then every point $X$ on the plane $P$ can be parametrized by
    	\[X(u,v) = \mathbf a u + \mathbf b v + \mathbf x_0\]
\end{enumerate}
\end{exmp}
\end{frame}

\begin{frame}
\begin{defn}
A $n\times m$ ($n$-by-$m$) \textbf{matrix} is a collection of $nm$ numbers (or functions) arranged in the following way.
	$$A= (a_{ij})_{n\times m}=\begin{pmatrix}
	a_{11} & a_{12} & \cdots & a_{1m} \\
	a_{21} & a_{22} & \cdots & a_{2m} \\
	\vdots & \vdots & & \vdots \\
	a_{n1} & a_{n2}& \cdots & a_{nm} \end{pmatrix}$$
The indices $i$, $j$ of an entry $a_{ij}$ represents the row and column indices respectively.
\end{defn}
\end{frame}

\begin{frame}
\begin{exmp}\*
\begin{enumerate}
	\item A $n\times m$ matrix is called \textbf{square} matrix if $n=m$.
	\item If $A$ is a square matrix and $a_{ij}=0$ for all $i\neq j$, then $A$ is called \textbf{diagonal}.
	$$A = \begin{pmatrix}
	a_{11} &  & \mathbf 0 \\
	& \ddots & \\
	\mathbf 0 & & a_{nn}
	\end{pmatrix}$$
	\item If the diagonal entries of a diagonal matrix are all $1$, then it is called the \textbf{identity matrix}, and denoted by $I_n$.
	$$I_n = \begin{pmatrix}
	1 &  & \mathbf 0 \\
	& \ddots & \\
	\mathbf 0 & & 1
	\end{pmatrix}$$
%	\item If $A$ is a square matrix and $a_{ij}=0$ for all $i<j$ ($i<j$, respectively), 
%	then $A$ is called \textbf{lower triangle} (\textbf{upper triangle}, respectively).
%	$$\begin{pmatrix}
%	\ast &  & \mathbf 0 \\
%	& \ddots & \\
%	\ast & & \ast
%	\end{pmatrix},\quad\begin{pmatrix}
%	\ast &  & \ast\\
%	& \ddots & \\
%	\mathbf 0 & & \ast
%	\end{pmatrix}$$
%	\item If every entry is $0$, then it is called the \textbf{zero matrix}, and denoted by $\mathbf 0$.
\end{enumerate}
\end{exmp}
\end{frame}

\begin{frame}
\begin{defn}
Let $A=(a_{ij})$ and $B=(b_{ij})$ be $n\times m$ matrix. Then we define
	\[A+B=(a_{ij}+b_{ij})\]
	\[k\cdot A=(ka_{ij})\]
Let $C$ be a $m\times l$ matrices, then $A\cdot B$ is a $n\times l$ matrix whose entries are
	\[A\cdot B=(a_{i1}b_{1j}+a_{i2}b_{2j}+\cdots+a_{im}b_{mj})_{1\le i\le n,1\le j\le l}\]
	$$=\begingroup\renewcommand*{\arraystretch}{.5}\begin{pmatrix}
	a_{11}  & a_{12} & \cdots & a_{1m} \\
	\vdots  & \vdots & & \vdots \\
	\tikzmark{left}a_{i1}  & a_{i2} & \cdots & a_{im}\tikzmark{right} \\
	\vdots & \vdots & & \vdots \\
	a_{n1}  & a_{n2} & \cdots & a_{nm} 
	\end{pmatrix}\endgroup\DrawBox[thick]
	\begingroup\renewcommand*{\arraystretch}{1.5}\begin{pmatrix}
	b_{11} \cdots \tikzmark{left}b_{1j} \cdots  b_{1l} \\
	b_{21}  \cdots b_{2j} \cdots  b_{2l} \\
	\vdots\,\quad\quad\vdots \quad\,\quad\vdots  \\
	b_{m1} \cdots  b_{mj}\tikzmark{right} \cdots  b_{ml}
	\end{pmatrix}\endgroup\DrawBox[thick]$$
\end{defn}
\end{frame}

\begin{frame}
\begin{exmp}
Suppose Bob, Larry, and Joanna worked in a fruits store for three days.
Table 1 shows \emph{how many} fruits each sold in total represented by the matrix $A$,
and Table 2 shows \emph{how much} was the fruits on each day represented by the matrix $B$.
\begin{table}[h!]
\centering
\begin{tabular}{lllll}
       & Apple & Orange & Banana &  \\
Bob    & 38    & 25     & 10     &  \\
Larry  & 15    & 22     & 15     &  \\
Joanna & 8     & 70     & 27     & 
\end{tabular}\caption{The volumn of sales of each person per items}
\end{table}
\begin{table}[h]
\centering
\begin{tabular}{lllll}
       & Day1   & Day2   & Day3   &  \\
Apple  & \$1.19 & \$1.45 & \$.99  &  \\
Orange & \$1.70 & \$0.99 & \$2.1  &  \\
Banana & \$2.19 & \$3.5  & \$1.29 & 
\end{tabular}\caption{The prices of items per day}
\end{table}
\end{exmp}
\end{frame}

\begin{frame}
\begin{exmp}
The $i,j$-entries of the $A\cdot B$ represents the total
revenue sold by the person $i$ at the day $j$. 

$$\begin{pmatrix}
	38 & 25 & 10  \\
	15 & 22 & 15  \\
	8 & 70 & 27  
	\end{pmatrix}\begin{pmatrix}
	1.19 & 1.45 & 0.99 \\
	1.70 & 0.99 & 2.1 \\
	2.19 & 3.5 & 1.29
    \end{pmatrix}$$
A deeper and important meaning of matrix multiplication
will be discovered when we visit the \emph{chain rule} in the latter section.
\end{exmp}
\end{frame}

\begin{frame}
\begin{prop}
Let $A,B,C$ be matrices. Whenever the operations are valid, the following holds.
	\begin{enumerate}
		\item $(A\cdot B)\cdot C=A\cdot(B\cdot C)$
		\item $A\cdot (B+C)=A\cdot B+A\cdot C$
		\item $(B+C)\cdot A=B\cdot A+C\cdot A$
		\item $k\cdot (A\cdot B)=(k\cdot A)\cdot B=A\cdot(k\cdot B)$
    \end{enumerate}
\end{prop}
\end{frame}

\begin{frame}
    \begin{prop}
The \textbf{transpose} of a matrix $A=(a_{ij})$ defined by
	\[A^T=(a_{ji})\]
and it satisfies the following.
	\begin{enumerate}\setcounter{enumi}{4}
		\item $\left(A^T\right)^T=A$
		\item $k\cdot A^T=(k\cdot A)^T$
		\item $(A+B)^T=A^T+B^T$
		\item $(AB)^T=B^TA^T$
	\end{enumerate}
\end{prop}
\end{frame}

\begin{frame}
\begin{defn}
The \textbf{determinant} of $2\times2$ matrix $A$ is defined as follows.
	\[\det A=\left|\begin{array}{cc}a_{11}&a_{12}\\a_{21}&a_{22}\end{array}\right|=a_{11}a_{22}-a_{12}a_{21}\]
The \textbf{determinant} of $3\times 3$ matrix $B$ is defined as follows.
	\begin{align*}
	\det B =\left|\begin{array}{ccc}b_{11}&b_{12}&b_{13}\\b_{21}&b_{22}&b_{23}\\b_{31}&b_{32}&b_{33}
	\end{array}\right|
	&=b_{11}b_{22}b_{33}+b_{12}b_{23}b_{31}+b_{13}b_{21}b_{32}\\
	&\quad -b_{13}b_{22}b_{31}-b_{11}b_{23}b_{32}-b_{12}b_{21}b_{33}
	\end{align*}
\end{defn}
\end{frame}

\begin{frame}
    \begin{defn}\label{defn-cross-prod}
Let $\mathbf a=(a_1,a_2,a_3)$, $\mathbf b=(b_1,b_2,b_3)$ be two $3$-dimensional vectors.
The \textbf{cross-product} of $\mathbf a,\mathbf b$ is a vector defined by
	\[\mathbf a\times\mathbf b=(a_2b_3-a_3b_2,a_3b_1-a_1b_3,a_1b_2-a_2b_1)\]
An easy way to remember the formula is the following.
	$$\mathbf a\times\mathbf b=
	\left|\begin{array}{ccc}
	\mathbf i & \mathbf j & \mathbf k \\
	a_1 & a_2 & a_3 \\
	b_1 & b_2 & b_3
	\end{array}\right|$$
\end{defn}
\end{frame}

\begin{frame}
\begin{prop}\label{prop-cross}
Let $\mathbf a,\mathbf b,\mathbf c$ be $3$-dimensional vectors and $k$ a constant. 
The following identity holds.
\begin{enumerate}
	\item $\mathbf a\times\mathbf 0=\mathbf 0\times\mathbf a=\mathbf 0$
	\item $\mathbf a\times\mathbf a=\mathbf 0$
	\item $\mathbf a\times\mathbf b=-\mathbf b\times \mathbf a$
	\item $\mathbf a\times (k\mathbf b)=(k\mathbf a)\times \mathbf b=k(\mathbf a\times\mathbf b)$
	\item $\mathbf a\times (\mathbf b+\mathbf c)=\mathbf a\times \mathbf b+\mathbf a\times\mathbf c$
	\item $\mathbf a\cdot(\mathbf a\times\mathbf b)=\mathbf b\cdot(\mathbf a\times\mathbf b)=0$ (This shows that the cross product $\mathbf a\times\mathbf b$ is normal to the plane spanned by $\mathbf a$ and $\mathbf b$.)
	\item $(\mathbf a\times\mathbf b)\times\mathbf c
	=(\mathbf a\cdot\mathbf c)\mathbf b-(\mathbf b\cdot\mathbf c)\mathbf a$
\end{enumerate}
\end{prop}
\end{frame}

\begin{frame}
\begin{thm}
Let $\mathbf a,\mathbf b$ be two $3$-dimensional vectors. Then
	\[\Vert \mathbf a\times\mathbf b\Vert =\Vert \mathbf a\Vert \cdot\Vert \mathbf b\Vert \cdot|\sin\theta|\]
where $\theta$ is the angle between $\mathbf a$ and $\mathbf b$.
\end{thm}
\end{frame}

\begin{frame}
\begin{prop}\label{prop-det-prop}
Let us denote $|A|$ by the determinant of a matrix $A$. Then the following holds
	\begin{enumerate}
		\item If $A$ has a row (or a column) whose entries are all zero, then $|A|=0$.
		\item Let $B$ be the matrix obtained by interchanging two rows (or columns) of $A$. Then $|B|=-|A|$.
		\item Let $B$ be a matrix obtained by multiply $c$ on a row (or column) followed by adding it to another row (or column).
		Then $|B|=|A|$.
	\end{enumerate}
\end{prop}
\end{frame}

\begin{frame}
\begin{defn}
Let $A$ be a $n\times n$ matrix. A matrix $B$ satisfying
	\[A\cdot B=B\cdot A=I_n\]
is called the \textbf{inverse of $A$}, denoted by $B=A^{-1}$.
If an inverse matrix $A^{-1}$ exists, then $A$ is said to be \textbf{non-singular}. 
Otherwise, it is called \textbf{singular}.
\end{defn}

\begin{thm}\label{thm-sing}
A matrix $A$ is singular if and only if $\det A=0$.
\end{thm}
\end{frame}

\begin{frame}
    \begin{prop}\label{prop-inv-mat} If $A$ is a $2\times 2$ matrix, then $A^{-1}$ is 
	$$A^{-1}=\frac{1}{\det A}\begin{pmatrix}
	a_{22} & -a_{12} \\ -a_{21} & a_{11} \end{pmatrix}$$
For a $3\times 3$-matrix $B$, the inverse is given by
	\begin{equation}\label{eqn-inv-3}
	B^{-1}=\frac{1}{\det B}\begin{pmatrix}
	c_{11} & -c_{21} & c_{31} \\
	-c_{12} & c_{22} & -c_{32} \\
	c_{13} & -c_{23} & c_{33}\end{pmatrix}
	\end{equation}
where each $c_{ij}$, called the \textbf{cofactor}, is the determinant of $2\times 2$-matrix obtained by deleting $i$th row and $j$th column. For example,
	$$c_{21}=\left|\begin{array}{ccc}
	b_{11}&b_{12}&b_{13}\\
	b_{21}\makebox(-10,0){\rule[1ex]{0.4pt}{3\normalbaselineskip}}&b_{22}\makebox(-6,0){\rule[1ex]{4.5\normalbaselineskip}{0.4pt}}&b_{23}\\
	b_{31}&b_{32}&b_{33}
	\end{array}\right|
	=\left|\begin{array}{cc}
	b_{12}&b_{13}\\
	b_{32}&b_{33}
	\end{array}\right|$$
Notice that the row and column indices are switched in \eqref{eqn-inv-3}.
\end{prop}
\end{frame}

\begin{frame}
    \begin{defn}\label{defn-abs-vec-sp}
A \textbf{vector space} $V$ is a set of element called \textbf{vectors} satisfying the following properties:
	\begin{enumerate}
	\item (Zero vector) $V$ contains the \textbf{zero vector} $\mathbf 0$, which is a unique vector satisfying
	$\mathbf 0 + \mathbf v = \mathbf v + \mathbf 0 = \mathbf v$ for all $\mathbf v\in V$.
	\item (Vector sum) For any two vectors $\mathbf v$, $\mathbf w\in V$, the vector $\mathbf v+\mathbf w$ lies in $V$.
	\item (Scalar multiplication) For any $k\in\mathbf R$ and $v\in V$, the vector $k\mathbf v$ lies in $V$.
	\end{enumerate}
A subset $V\subset\mathbf R^n$ is called a \textbf{vector subspace} if it is a vector space itself.
\end{defn}
\end{frame}

\begin{frame}
    \begin{exmp}
	Let $\mathcal C^1(\mathbf R)$ be the set of all differentiable functions on $\mathbf R$
	whose derivatives are continuous on $\mathbf R$. Since $f,g$ are such functions so is the function $h=f+g$.
	Also for any $k\in\mathbf R$, the function $kf$ is differentiable and its derivative is continuous.
    Thus $\mathcal C^1(\mathbf R)$ is a vector space. Likewise, we can define vectors spaces $\mathcal C^n(\mathbf R)$, $\mathcal C^\infty(\mathbf R)$
    \end{exmp}
\end{frame}

\begin{frame}
\begin{exmp}
	For each constant $k\in\mathbf R$,
	let $V_k$ be the set of all points on the line $y=kx$ in $\mathbf R^2$:
		\[V_k = \{(x,y)\,|\,y=kx\}\]
	Then $V_k$ is a vector subspace of $\mathbf R^2$.
	Let $V_\infty$ be the vertical line $V_\infty = \{(0,y)\,|\,y\in\mathbf R\}$.
    Then $V_\infty$ is also a vector subspace of $\mathbf R^2$
\end{exmp}
\end{frame}

\begin{frame}
    \begin{exmp}
	Let $P$ be the set of all point on the plane 
		\[P = \{(x,y,z)\,|\,ax+by+cz=0\}\]
	in $\mathbf R^3$. By identifying points in $P$ as position vectors, we can say $P$ is a vector 
	subspace of $\mathbf R^3$, orthogonal to $\mathbf n=(a,b,c)$ 
	
	In particular, let $V,W$ be vector subspace of $\mathbf R^n$. Then so is $V\cap W$. 
	For example, let $V,W$ be vector subspace for two planes in $\mathbf R^3$ 
	passing through the origin $\mathbf 0$. Then $V\cap W$ is either a line (if $V,W$ are transversal) 
	or a plane (if $V=W$).
\end{exmp}
\end{frame}

\begin{frame}
    \begin{defn}
Let $V$ a vector space and $\mathbf v_1,\cdots,\mathbf v_m\in V$.
We say the vector space subspace 
	\[W = \{a_1\mathbf v_1+\cdots +a_m\mathbf v_m\,|\,a_1,\cdots, a_m\in\mathbf R\}\]
is \textbf{spanned} by $\mathbf v_1,\cdots,\mathbf v_m$,
and denote by $W=\textup{span}\langle\mathbf v_1,\cdots,\mathbf v_m\rangle$.
\end{defn}
\end{frame}

\begin{frame}
\begin{defn}
We say a vector space $V$ is \textbf{spanned by} the set of vectors $\{\mathbf v_1,\cdots,\mathbf v_n\}$,
if every $v\in V$ can be expressed a linear combination of $v_i$'s:
	\[v = a_1\mathbf v_1+\cdots a_n\mathbf v_n\]
The set $\{\mathbf v_1,\cdots,\mathbf v_n\}$ is said to be \textbf{linearly independent}
if the coefficients satisfying
	\[a_1\mathbf v_1+\cdots a_n\mathbf v_n=\mathbf 0\]
is the trivial ones, namely $a_1=\cdots=a_n=0$.
\end{defn}


\begin{defn}
Let $\{\mathbf v_1,\cdots,\mathbf v_n\}$ be the set of linear independent vectors which spans the vector space $V$. 
Such set is called the \textbf{basis} of $V$, and its element is called the \textbf{basis vector}.
The \textbf{dimension} of $V$ is the number of basis vectors which spans $V$.
\end{defn}
\end{frame}

\begin{frame}
\begin{prob}
\begin{enumerate}
	\item Show that the vectors $\Vert\mathbf b\Vert\mathbf a +\Vert \mathbf a\Vert\mathbf b$
	and $\Vert\mathbf b\Vert\mathbf a-\Vert\mathbf a\Vert\mathbf b$ are orthogonal.
	\item Show that $\Vert \mathbf b\Vert \mathbf a +\Vert \mathbf a\Vert\mathbf b$ bisects the angle between $\mathbf a$ and $\mathbf b$.
\end{enumerate}
\end{prob}
\end{frame}

\begin{frame}
\begin{prob}
\begin{enumerate}
\item Let $\mathbf a=(1,2,1)$, $\mathbf b=(2,1,2)$, and $\mathbf u=(0,1,-1)$.
Suppose that the vector $\mathbf u$ can be decomposed by
	\[\mathbf u = \mathbf u_1 + \mathbf u_2 + \mathbf u_3\]
where $\mathbf u_1$ is parallel to $\mathbf a$, $\mathbf u_2$ is parallel to $\mathbf b$, and
$\mathbf u_3$ is orthogonal to both $\mathbf a$ and $\mathbf b$.
Find the vector $\mathbf u_1$, $\mathbf u_2$, $\mathbf u_3$ explicitly.
\end{enumerate}
\end{prob}
\end{frame}

\begin{frame}
\begin{prob}
\begin{enumerate}
\item Find the distance between $P=(2,1,3)$ and the line $l(t) = (2,3,-2)+t(-1,1,-2)$.
\item Find the distance between two parallel planes
	\[2x-2y+z=5,\quad 2x-2y+z=20\]
\item Find the distance between two skew lines
	\[l_1(t) = (0,5,-1) + t(2,1,3)\]
	\[l_2(t) = (-1,2,0) + t(1,-1,0)\]
\end{enumerate}
\end{prob}
\end{frame}

\begin{frame}
\begin{prob}
Let $\mathbf a = (a_1,a_2,a_3)$, $\mathbf b = (b_1,b_2,b_3)$, $\mathbf c = (c_1,c_2,c_3)$.
Verify that
$$(\mathbf a\times\mathbf b)\cdot\mathbf c =
	\left|\begin{array}{ccc}
	a_1 & a_2 & a_3 \\
	b_1 & b_2 & b_3 \\
	c_1 & c_2 & c_3
	\end{array}\right|$$
\end{prob}
\end{frame}

\begin{frame}
\begin{prob}
Imagine two concentric circles with radius $a,b$ ($b<a$) which rolls on flat line with the same angular velocity.
A curtate cycloid is a trajectory of a point on the circle of radius $b$.
Find a set of parametric equation for the curtate cycloid with $a=3$, $b=2$.
\end{prob}
\end{frame}

\end{document}

