\documentclass{beamer}
\usefonttheme{serif}

\usepackage{amsmath,amsthm,amssymb,amsfonts,amscd,mathrsfs,amsxtra,multirow,kotex,mathtools,gensymb,textcomp,lipsum,tikz,verbatim,color,soul,courier,mdframed,xcolor}
\usepackage[normalem]{ulem}
\usetikzlibrary{calc,matrix,arrows,chains,positioning,scopes}
\usepackage{pdfpages}

\theoremstyle{plain}
\newtheorem{thm}{Theorem}[section]
\newtheorem{prop}[thm]{Proposition}

\theoremstyle{definition}
\newtheorem{defn}[thm]{Definition}
\newtheorem{exmp}[thm]{Example}
\newtheorem{excs}[thm]{Exercise}
\newtheorem{rem}[thm]{Remark}
\newtheorem{prob}[thm]{Problem}

\newcommand \tr[1]{\textcolor{red}{#1}}
\newcommand{\tikzmark}[1]{\tikz[overlay,remember picture] \node (#1) {};}
\newcommand{\varep}{\varepsilon}
\newcommand{\DrawBox}[1][]{%
    \tikz[overlay,remember picture]{
    \draw[red,#1]
      ($(left)+(-0.2em,0.9em)$) rectangle
      ($(right)+(0.2em,-0.3em)$);}
}

\newcommand{\tikzmarkk}[2]{
    \tikz[overlay,remember picture,baseline] 
    \node[anchor=base] (#1) {$#2$};
}
\newcommand*\circled[1]{\tikz[baseline=(char.base)]{
            \node[shape=circle,draw,inner sep=2pt] (char) {#1};}}

\tikzset{join/.code=\tikzset{after node path={%
\ifx\tikzchainprevious\pgfutil@empty\else(\tikzchainprevious)%
edge[every join]#1(\tikzchaincurrent)\fi}}}

\tikzset{>=stealth',every on chain/.append style={join},
         every join/.style={->}}
\tikzstyle{labeled}=[execute at begin node=$\scriptstyle,
   execute at end node=$]

\newenvironment<>{proofs}[1][\proofname]{%
   \par
   \def\insertproofname{#1\@{.}}%
   \usebeamertemplate{proof begin}#2}
 {\usebeamertemplate{proof end}}
 

\addtobeamertemplate{navigation symbols}{}{%
    \usebeamerfont{footline}%
    \usebeamercolor[fg]{footline}%
    \hspace{1em}%
    \raisebox{2pt}[0pt][0pt]{\insertframenumber/\inserttotalframenumber}
}
\setbeamercolor{footline}{fg=blue}
\setbeamerfont{footline}{series=\bfseries}
\title[]{SE102:Multivariable Calculus}

\author[]{Hyosang Kang\inst{1}}

\institute[]{\inst{1}Division of Mathematics\\ School of Interdisciplinary Studies\\ DGIST}

\date[]{Week 03}

\begin{document}

\begin{frame}
\titlepage
\end{frame}

\begin{frame}
\begin{thm}[The chain rule]
    Let $f(x,y)$ be a two variable function 
    and $c(t) = (x(t),y(t))$ be a parametrization 
    of a curve in $\mathbf R^2$ 
    which is differentiable at $t_0$. 
    If $f(x,y)$ is differentiable at 
    $(x_0,y_0)=(x(t_0),y(t_0))$,
    then the composition 
    	$$F(t)=(f\circ c)(t) = f(x(t),y(t))$$
    is also differentiable at $t_0$ 
    and its differential is 
    $$F'(t_0)=f_x(x_0,y_0)x'(t_0)+f_y(x_0,y_0)y'(t_0)$$
\end{thm}
\end{frame}

\begin{frame}
\begin{proofs}
    Let us define $g_1(t),g_2(t)$ as follows.
	$$g_1(t) = \frac{x(t)-x_0-x'(t_0)(t-t_0)}{t-t_0}$$
	$$g_2(t) = \frac{y(t)-y_0-y'(t_0)(t-t_0)}{t-t_0}$$
    Since $x(t)$, $y(t)$ are differentiable at $t_0$,
	$$\lim_{t\to t_0}g_1(t) = \lim_{t\to t_0}g_2(t)=0.$$
    Thus $g_1(t)$, $g_2(t)$ are continuous.
    Define $F(x,y)$ as
	$$F(x,y) = \frac{f(x,y)-L(x,y)}{\sqrt{(x-x_0)^2+(y-y_0)^2}}.$$
\end{proofs}
\end{frame}

\begin{frame}
\begin{proofs}
    Since we assumed that $f(x,y)$ is differentiable 
    at $(x_0,y_0)$,
	$$\lim_{(x,y)\to(x_0,y_0)}F(x,y) = 0.$$
    Thus $F(x,y)$ is continuous on $\mathbf R^2$.
    Note that
	$$x(t)-x_0 = (x'(t_0) + g_1(t))(t-t_0),$$
	$$y(t)-y_0 = (y'(t_0) + g_2(t))(t-t_0),$$
    and we can rewrite the definition of $F(x,y)$ as
    \begin{align*}
    f(x,y)-f(x_0,y_0)
    &= f_x(x_0,y_0)(x-x_0) + f_y(x_0,y_0)(y-y_0)\\
    &\quad + F(x,y)\sqrt{(x-x_0)^2+(y-y_0)^2}.
    \end{align*}
\end{proofs}
\end{frame}

\begin{frame}
\begin{proofs}
    With further computation, we get
	\begin{align*}
	&\frac{f(x(t),y(t))-f(x_0,y_0)}{t-t_0} \\
	&\quad = \frac{1}{t-t_0}\Big(
        f_x(x_0,y_0)(x(t)-x_0) 
        + f_y(x_0,y_0)(y(t)-y_0) \\
    &\qquad\qquad + F(x(t),y(t))
        \sqrt{(x(t)-x_0)^2+(y(t)-y_0)^2}\Big)\\
	&\quad = f_x(x_0,y_0)(x'(t_0) + g_1(t)) + f_y(x_0,y_0)(y'(t_0) + g_2(t)) \\
	&\qquad\qquad + F(x(t),y(t))\sqrt{\left(\frac{x(t)-x_0}{t-t_0}\right)^2+\left(\frac{y(t)-y_0}{t-t_0}\right)^2} \\
	\end{align*}
\end{proofs}
\end{frame}

\begin{frame}
\begin{proofs}
Let us take the limit on both sides:
    \begin{align*}
    &\lim_{t\to t_0}\frac{f(x(t),y(t))-f(x_0,y_0)}{t-t_0}\\
	&\quad = f_x(x_0,y_0)x'(t_0) + f_y(x_0,y_0)y'(t_0) \\
    &\qquad + f_x(x_0,y_0)\lim_{t\to t_0}g_1(t) + f_y(x_0,y_0)\lim_{t\to t_0}g_2(t) \\
    &\qquad + \lim_{t\to t_0}F(x(t),y(t))
     \sqrt{\left(\lim_{t\to t_0}\frac{x(t)-x_0}{t-t_0}\right)^2
     + \left(\lim_{t\to t_0}\frac{y(t)-y_0}{t-t_0}\right)^2}\\
    &\quad = f_x(x_0,y_0)x'(t_0) + f_y(x_0,y_0)y'(t_0).
    \end{align*}
This proves the theorem.
\end{proofs}
\end{frame}

\begin{frame}
\begin{rem}
We can write the chain rule using the gradient of $f(x,y)$.
$$(f\circ c)'(t_0) = \nabla f(c(t_0))\cdot c'(t_0)$$
The formula says that
\emph{the rate of change of $f(x,y)$ at $(x_0,y_0)$ 
to the direction of $c'(t_0)$ 
is given by the inner product.}\\
Thus we can say that 
\emph{the gradient measures how much $f(x,y)$ changes 
to the given direction.}
\end{rem}
\end{frame}

\begin{frame}
\begin{rem}
We can define the gradient $\nabla f(x_0,y_0)$ 
as a linear map $\nabla f(x_0,y_0): \mathbf R^2 \to \mathbf R$
such that 
$$\nabla f(x_0,y_0)(\mathbf v+\mathbf w) 
= \nabla f(x_0,y_0)\cdot\mathbf v 
  + \nabla f(x_0,y_0)\cdot\mathbf w$$
for all $\mathbf v$, $\mathbf w$ in $\mathbf R^2$.
We will see that this is a particular case of /emph{differential}.
\end{rem}
\end{frame}

\begin{frame}
\begin{defn}
Let $F:\mathbf R^2\to\mathbf R^2$ be a two-variable function
whose coordinates are defined as follows:
    $$F(u,v) = (x(u,v), y(u,v))$$
We say $F$ is continuous (differentiable, repectively) 
at $(u_0, v_0)$ if all coordinate functions 
$x, y:\mathbf R^2\to\mathbf R$ are continuous 
(differentiable, respectively) at $(u_0,v_0)$. 
\end{defn}
\end{frame}

\begin{frame}
\begin{thm}
Let $\mathbf X:\mathbf R^2\to\mathbf R^2$ be a function 
defined as 
$$\mathbf X(u,v) = (x(u,v),y(u,v)).$$
Let us denote $x_0=x(u_0,v_0)$, $y_0=y(u_0,v_0)$.
If $X$ are differentiable at $(u_0,v_0)$,
then for a function $f:\mathbf R^2\to\mathbf R$ 
which is differentiable at $(x_0,y_0)$,
the composite function $F = f\circ X$ 
is also differentiable at $(u_0,v_0)$.
Moreoever, 
\begin{align*}
    F_u(u_0,v_0) &= f_x(x_0,y_0) x_u(u_0,v_0) + f_y(x_0,y_0) y_u(u_0,v_0),\nonumber\\
	F_v(u_0,v_0) &= f_x(x_0,y_0) x_v(u_0,v_0) + f_y(x_0,y_0) y_v(u_0,v_0)
\end{align*}
\end{thm}
\end{frame}

\begin{frame}
\begin{proofs}
If we assume that $F=f\circ X$ is differentiable,
then by the chain rule that we proved earlier, we get
	\begin{align*}
    F_u(u_0,v_0) 
        &= \frac{d}{du}\Big|_{u=u_0}f(x(u,v_0),y(u,v_0)) \\
        &= f_x(x_0,y_0)x_u(u_0,v_0) + f_y(x_0,y_0)y_u(u_0,v_0),\\
	F_v(u_0,v_0) 
        &= \frac{d}{dv}\Big|_{v=v_0}f(x(u_0,v),y(u_0,v)) \\
        &= f_x(x_0,y_0)x_v(u_0,v_0) + f_y(x_0,y_0)y_v(u_0,v_0)
    \end{align*}
Let us define  
    \begin{align*}
    S(u,v) &= F(u,v) - F(u_0,v_0) \\
        &\qquad - \begin{bmatrix}f_x(x_0,y_0) & f_y(x_0,y_0)\end{bmatrix}
          \begin{bmatrix}x_u & x_v \\ y_u & y_v\end{bmatrix}
          \begin{bmatrix}u-u_0 \\ v-v_0\end{bmatrix}
    \end{align*}
\end{proofs}
\end{frame}

\begin{frame}
\begin{proofs}
It suffices to show that 
$\displaystyle\lim_{(u,v)\to(u_0,v_0)}
\frac{S(u,v)}{\sqrt{(u-u_0)^2+(v-v_0)^2}}=0$.
Let us define
\begin{align*}
H(u,v) = \frac{X(u,v)-X(u_0,v_0) 
    - \begin{bmatrix}x_u & x_v \\ y_u & y_v\end{bmatrix}
      \begin{bmatrix}u-u_0 \\ v-v_0\end{bmatrix}}
    {\sqrt{(u-u_0)^2+(v-v_0)^2}}.
\end{align*}
From the differentiability of $X$, we have
$\displaystyle\lim_{(u,v)\to(u_0,v_0)}H(u,v) = 0$.
Note that $\displaystyle X(u,v) = \begin{bmatrix}x\\y\end{bmatrix}$,
$X(u_0,v_0) = \begin{bmatrix}x_0 \\y_0\end{bmatrix}$.
Thus
$$\begin{bmatrix}x_u & x_v \\ y_u & y_v\end{bmatrix}
  \begin{bmatrix}u-u_0 \\ v-v_0\end{bmatrix}
  = \begin{bmatrix}x-x_0\\y-y_0\end{bmatrix}
    - H(u,v)\sqrt{(u-u_0)^2+(v-v_0)^2}$$
\end{proofs}
\end{frame}

\begin{frame}
\begin{proofs}
By substituting above into the formula of $S(u,v)$, we get
\begin{align*}
S(u,v) 
    &= f(x,y) - f(x_0,y_0) -
        \begin{bmatrix}f_x(x_0,y_0) & f_y(x_0,y_0)\end{bmatrix}
        \begin{bmatrix}x-x_0\\y-y_0\end{bmatrix} \\
    &\qquad + \begin{bmatrix}f_x(x_0,y_0) & f_y(x_0,y_0)\end{bmatrix}
               H(u,v)\sqrt{(u-u_0)^2+(v-v_0)^2}
\end{align*}
Then,
\begin{align*}
&\frac{S(u,v)}{\sqrt{(u-u_0)^2+(v-v_0)^2}}\\
&=\frac{f(x,y)-f(x_0,y_0)-
    \begin{bmatrix}f_x(x_0,y_0) & f_y(x_0,y_0)\end{bmatrix}
    \begin{bmatrix}x-x_0\\y-y_0\end{bmatrix}}
        {\sqrt{(u-u_0)^2+(v-v_0)^2}} \\
&\qquad + \begin{bmatrix}f_x(x_0,y_0) & f_y(x_0,y_0)\end{bmatrix}H(u,v)
\end{align*}
\end{proofs}
\end{frame}

\begin{frame}
\begin{proofs}
By the differentiability of $X$, 
the second summand vanishes as $(u,v)\to(u_0,v_0)$.
We can rewrite the first summand as multiple of two factors below:
\begin{equation}\label{eq-fac-1}
\frac{f(x,y)-f(x_0,y_0)-
\begin{bmatrix}f_x(x_0,y_0) & f_y(x_0,y_0)\end{bmatrix}
\begin{bmatrix}x-x_0\\y-y_0\end{bmatrix}}
    {\sqrt{(x-x_0)^2+(y-y_0)^2}}
\end{equation}
\begin{equation}\label{eq-fac-2}
\frac{\sqrt{(x-x_0)^2+(y-y_0)^2}}{\sqrt{(u-u_0)^2+(v-v_0)^2}}
\end{equation}
\eqref{eq-fac-1} vanishes as $(x,y)\to(x_0,y_0)$ due to the differentiability of $f$.
Thus it suffices to prove that \eqref{eq-fac-2} is bounded.
Note that $|X(u,v)-X(u_0,v_0)|=\sqrt{(x-x_0)^2+(y-y_0)^2}$.
Thus \eqref{eq-fac-2} satisfies
$$\frac{\sqrt{(x-x_0)^2+(y-y_0)^2}}{\sqrt{(u-u_0)^2+(v-v_0)^2}}
\le |H(u,v)| + \left|
\begin{bmatrix}x_u & x_v \\ y_u & y_v\end{bmatrix}
\begin{bmatrix}u-u_0 \\ v-v_0\end{bmatrix}\right|$$
\end{proofs}
\end{frame}

\begin{frame}
\begin{proofs}
Again, $H(u,v)$ is vanishes as $(u,v)\to(u_0,v_0)$.
By the inequality
$$ \left|\begin{bmatrix}x_u & x_v \\ y_u & y_v\end{bmatrix}
\begin{bmatrix}u-u_0 \\ v-v_0\end{bmatrix}\right|
\le \sqrt2\max\{|x_u|,|x_v|,|y_u|,|y_v|\}\left|
\begin{bmatrix}u-u_0 \\ v-v_0\end{bmatrix}\right|,$$
we are done.
\end{proofs} 
\end{frame}

\begin{frame}
\begin{rem}
    We can write the chain rule using matrices.
    For $F(t) = (f\circ c)(t)$, 
    $$F'(t) = \begin{bmatrix} f_x & f_y \end{bmatrix}
        \cdot\begin{bmatrix} x'(t) \\ y'(t) \end{bmatrix}$$
    For $F(u,v) = (f\circ X)(u,v)$, 
	$$\begin{bmatrix} F_u & F_u	\end{bmatrix}=
        \begin{bmatrix}	f_x & f_y \end{bmatrix}
        \cdot \begin{bmatrix} x_u & x_v \\ y_u & y_v \end{bmatrix}.$$
    Note that the gradient of $F$ and $f\circ X$ can be viewed as
    $1\times 2$ matrices. 
    Thus the last equation can be written as
    $$\nabla(f\circ X) = \nabla f \cdot DX$$
\end{rem}
\end{frame}

\begin{frame}
\begin{exmp}
    Let $T(r,\theta) = (r\cos\theta,r\sin\theta)$. 
    Suppose that $F = f\circ T$. Then
	$$F_r = f_x x_r + f_y y_r = f_x \cos\theta + f_y\sin\theta$$
	$$F_\theta = f_x x_\theta + f_y y_\theta = -rf_x\sin\theta + r f_y\cos\theta$$
    In matrix form,
    $$\begin{bmatrix} F_r & F_\theta \end{bmatrix}
     = \begin{bmatrix} f_x & f_y \end{bmatrix}
       \cdot \begin{bmatrix} \cos\theta & -r\sin\theta \\
             \sin\theta & r\cos\theta \end{bmatrix}.$$
\end{exmp}
\end{frame}

\begin{frame}
\begin{defn}
Let $\mathbf f:\mathbf R^n\to\mathbf R^m$ be a differentiable function.
Then the \textbf{differential} at $\mathbf a\in\mathbf R^n$
is the $m\times n$ matrix $\mathbf{df}(\mathbf a)$
defined as below:
$$\mathbf{df}(\mathbf a) = \begin{bmatrix}
    \frac{\partial f_1}{\partial x_1}(\mathbf a) & \cdots & \frac{\partial f_1}{\partial x_n}(\mathbf a) \\
    \vdots & \ddots & \vdots \\
    \frac{\partial f_m}{\partial x_1}(\mathbf a) & \cdots & \frac{\partial f_m}{\partial x_n}(\mathbf a)
\end{bmatrix}$$
Here, each $f_i$ is the coordinate function for $f$: 
$$f(\mathbf x) = (f_1(\mathbf x),\cdots,f_n(\mathbf x)).$$
\end{defn}
\end{frame}

\begin{frame}
\begin{thm}[Chain rule]
    Let $\mathbf f:\mathbf R^n\to\mathbf R^m$ be
    a differentiable function at $\mathbf a\in \mathbf R^n$ 
    and $\mathbf g:\mathbf R^m\to\mathbf R^l$ be
    a differentiable function at $\mathbf f(\mathbf a)\in\mathbf R^m$. 
    Then $\mathbf g\circ\mathbf f$ is a differentiable function
    at $\mathbf a$, and
    $$\mathbf{d( g\circ f)}(\mathbf a) 
    = \mathbf{dg}(\mathbf f(\mathbf a))\mathbf{df}(\mathbf a)$$
\end{thm}
\end{frame}

\begin{frame}
\begin{defn}
Let $S$ be a subset of $\mathbf R^n$.
The \textbf{tangent space} $T_{\mathbf a}S$ at $\mathbf a\in S$
is the vector space consists of all \emph{tangent} vectors of 
$S$ at $\mathbf a$.
\end{defn}
\end{frame}

\begin{frame}
\begin{rem}
    The tangent space $T_{\mathbf a}\mathbf R^n$ 
    is a $n$-dimensional vector space $\mathbf R^n$.
    Sicne $\mathbf{df}(\mathbf a)$ is a matrix, 
    one can view the differential $\mathbf{df}(\mathbf a)$ 
    as a linear map 
    $\mathbf{df}(\mathbf a):\mathbf R^n\to\mathbf R^m$
    as follows: 
    for each $n$-dimensional vector 
    $\mathbf v = (v_1,\cdots, v_n)$ in $T_{\mathbf a}\mathbf R^n$,
    the vector $\mathbf{df}(\mathbf a)\mathbf v$ 
    is a $m$-dimensional vector defined by
    $$d\mathbf f(\mathbf a)\mathbf v 
    = \begin{bmatrix}
    \frac{\partial f_1}{\partial x_1}(\mathbf a) 
        & \cdots &\frac{\partial f_1}{\partial x_n}(\mathbf a) \\
	\vdots & \ddots & \vdots \\
    \frac{\partial f_m}{\partial x_1}(\mathbf a) 
        & \cdots &\frac{\partial f_m}{\partial x_n}(\mathbf a) 
    \end{bmatrix}
    \begin{bmatrix} v_1 \\ \vdots \\ v_n \end{bmatrix}$$
\end{rem}
\end{frame}

\begin{frame}
\begin{exmp}
Let $f(x,y,z) = x+2y+3z$.
Let $S$ be the graph of $z=xy$. 
The differential $df(p)$ at $p=(1,1,1)$ on $S$ is
	$$df(1,1,1) = \begin{bmatrix} 1 & 2 & 3 \end{bmatrix}.$$
The tangent plane of $S$ at $p$ is
	$$x+y-z=1$$
and the tangent space $T_pS$ is spanned by 
	$$\mathbf v_1=(1,0,1),\quad\mathbf v_2=(0,1,1).$$
\end{exmp}
\end{frame}

\begin{frame}
The parametric curve $c(t) = (t,1,t)$ lies on $S$, 
Since $c(1)=p$ and $c'(1)=\mathbf v_1$, 
by the chain rule, 
    $$(f\circ c)'(1)
    = \begin{bmatrix} 1 & 2 & 3 \end{bmatrix} 
      \begin{bmatrix} 1 \\ 0 \\ 1 \end{bmatrix} = 4$$
In other words, $df(p)\cdot \mathbf v_1 = 4$.
Similarly, we can compute $df(p)\cdot \mathbf v_2 = 5$.
Since given a tangent vector 
$\mathbf v = a\mathbf v_1+b\mathbf v_2$ for suitable $a,b$,
the rate of change of $f$ in direction of $\mathbf v$ is
    $$df(p)\cdot\mathbf v = 4a + 5b.$$
\end{frame}

\begin{frame}
\begin{prob}
A cylinderical ice is melting in a room.
When the radius of ice is $6$(cm) and the height is $10$(cm),
the radius is decreasing at $0.1$(cm/min) and 
the height is decreasing at $0.2$(cm/min).
How fast (cm\^{}3/min) is the ice melting?
\end{prob}
\end{frame}

\begin{frame}
\begin{prob}
    Let $f(x,y) = \sqrt{|xy|}$.
    Find the tangent plane to the graph at $(1,1,1)$.
\end{prob}
\end{frame}

\begin{frame}
\begin{prob}
    Let $f(x,y) = (x^2-y^2, 2xy)$.
    \begin{enumerate}
        \item Find the differential $df(1,1)$.
        \item Let $D = [1,1+\varepsilon]\times[1,1+\varepsilon]$.
        Compute the limit
            $$\lim_{\varepsilon\to0}\frac{\textrm{area}f(D)}{\textrm{area}(D)}$$
        \item Find any relation between results in 1 and 2.
    \end{enumerate}
\end{prob}
\end{frame}

\begin{frame}
\begin{prob}
Let $f(x,y)$ be a function with continuous partial derivatives.
Let $x = e^r\cos\theta$ and $y = e^r\sin\theta$. Show that
    $$\left(\frac{\partial f}{\partial x}\right)^2 
    + \left(\frac{\partial f}{\partial y}\right)^2 
    = e^{-2r}\left(\left(\frac{\partial f}{\partial r}\right)^2
    + \left(\frac{\partial f}{\partial\theta}\right)^2\right)$$
\end{prob}
\end{frame}

\begin{frame}
\begin{prob}
    The \emph{Laplacian} $\Delta f$ of $f(x,y)$ is defined as
    $$\Delta f = \frac{\partial^2f}{\partial x^2}
     + \frac{\partial^2f}{\partial y^2}$$
	Show that 
    $$\Delta f = \frac{\partial^2 f}{\partial r^2} 
     + \frac{1}{r}\frac{\partial f}{\partial r} 
     + \frac{1}{r^2}\frac{\partial^2f}{\partial \theta^2}$$
\end{prob}
\end{frame}
\end{document}